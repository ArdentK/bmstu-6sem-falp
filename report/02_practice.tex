\chapter{Практические задания}
\begin{enumerate}[wide=0pt]
	\item \textit{Написать хвостовую рекурсивную функцию my-reverse, которая развернет верхний уровень своего списка-аргумента lst.}
	\begin{lstlisting}
(defun my-reverse (lst &optional (buf-lst Nil))
	(cond ((null lst) buf-lst)
		(t (my-reverse (cdr lst) (cons (car lst) buf-lst)))))
	\end{lstlisting}
	\item \textit{Написать функцию, которая возвращает первый элемент списка -аргумента, который сам является непустым списком.}
	\begin{lstlisting}
(defun not-null-lst (lst)
     (cond ((null lst) Nil)
           ((and (listp (car lst)) (not (null (caar lst)))) (car lst))
           (t (not-null-lst (cdr lst)))))
	\end{lstlisting}
	\item \textit{Написать функцию, которая выбирает из заданного списка только те числа, которые больше 1 и меньше 10. (Вариант: между двумя заданными границами. )}
	\begin{lstlisting}	
(defun append-elem (lst elem &optional (before-lst Nil))
	(cond ((and (null lst) (null before-lst)) (cons elem Nil))
		((null lst) (my-reverse (cons elem before-lst)))
		(t (append-elem (cdr lst) elem (cons (car lst) before-lst)))))

(defun append-lst (lst1 lst2 &optional (before-lst Nil))
	(cond ((and (null lst1) (null lst2)) before-lst)
		((not (null lst1)) (append-lst (cdr lst1) lst2 (append-elem before-lst (car lst1))))
		(t (append-lst lst1 (cdr lst2) (append-elem before-lst (car lst2))))))

(defun get-between (num1 num2 lst &optional (res-lst Nil))
	(cond ((null lst) res-lst)
		((and (numberp (car lst)) (< num1 (car lst) num2)) (get-between num1 num2 (cdr lst) 
		(append-elem res-lst (car lst))))
        (t (get-between num1 num2 (cdr lst) res-lst))))
	\end{lstlisting}
	\item \textit{Напишите рекурсивную функцию, которая умножает на заданное число-аргумент все числа из заданного списка-аргумента, когда 
		a) все элементы списка --- числа,
		6) элементы списка -- любые объекты.}
	\lstinputlisting[firstline=28, lastline=37]{../labs/7.lisp}
	\item \textit{Напишите функцию, select-between, которая из списка-аргумента, содержащего только числа, выбирает только те, которые расположены между двумя указанными границами-аргументами и возвращает их в виде списка (упорядоченного по возрастанию списка чисел (+ 2 балла)).}
	\lstinputlisting[firstline=40, lastline=56]{../labs/7.lisp}
	\item \textit{Написать рекурсивную версию (с именем rec-add) вычисления суммы чисел заданного списка: а) одноуровнего смешанного, б) структурированного.}
	\lstinputlisting[firstline=59, lastline=69]{../labs/7.lisp}
	\item \textit{Написать рекурсивную версию с именем recnth функции nth.}
	\lstinputlisting[firstline=71, lastline=74]{../labs/7.lisp}
	\item \textit{Написать рекурсивную функцию allodd, которая возвращает t когда все элементы списка нечетные.}
	\lstinputlisting[firstline=77, lastline=81]{../labs/7.lisp}
	\item \textit{Написать рекурсивную функцию, которая возвращает первое нечетное число из списка (структурированного), возможно создавая некоторые вспомогательные функции.}
	\lstinputlisting[firstline=84, lastline=90]{../labs/7.lisp}
	\item \textit{Используя cons-дополняемую рекурсию с одним тестом завершения, написать функцию которая получает как аргумент список чисел, а возвращает список квадратов этих чисел в том же порядке.}
	\lstinputlisting[firstline=93, lastline=95]{../labs/7.lisp}
\end{enumerate}